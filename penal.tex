CÓDIGO PENAL

    Santiago, noviembre 12 de 1874.

    Núm. 2561.-Para los efectos de la ley de esta fecha, en que se aprueba el Código Penal, nombro una comisión compuesta del Oficial Mayor del Ministerio de Justicia, Culto e Instrucción Pública, don Carlos Riesco, y de los jefes de sección del mismo Ministerio don Manuel Egidio Ballesteros y don Ramón C. Briseño.

    Anótese.

    ERRÁZURIZ.

    JOSE MARIA BARCELÓ


    Certificamos que la presente edición del Código Penal está conforme con el proyecto aprobado por el Congreso Nacional.-Santiago, diciembre 15 de 1874.

    CARLOS RIESCO.-M. E. BALLESTEROS.-RAMON C. BRISEÑO.


    EL PRESIDENTE DE LA REPÚBLICA.

    Santiago, noviembre 12 de 1874.

    Por cuanto el Congreso Nacional ha aprobado el siguiente

    CÓDIGO PENAL.

    LIBRO PRIMERO





    TÍTULO PRIMERO.

DE LOS DELITOS Y DE LAS CIRCUNSTANCIAS QUE EXIMEN DE RESPONSABILIDAD CRIMINAL, LA ATENÚAN O LA AGRAVAN.





    § I.

    De los delitos.



    ARTÍCULO 1.

    Es delito toda acción u omisión voluntaria penada por la ley.
    Las acciones u omisiones penadas por la ley se reputan siempre voluntarias, a no ser que conste lo contrario.

    El que cometiere delito será responsable de él e incurrirá en la pena que la ley señale, aunque el mal recaiga sobre persona distinta de aquella a quien se proponía ofender. En tal caso no se tomarán en consideración las circunstancias, no conocidas por el delincuente, que agravarían su responsabilidad; pero sí aquellas que la atenúen.


    ART. 2.

    Las acciones u omisiones que cometidas con dolo o malicia importarían un delito, constituyen cuasidelito si sólo hay culpa en el que las comete.



    ART. 3.

    Los delitos, atendida su gravedad, se dividen en crímenes, simples delitos y faltas y se califican de tales según la pena que les está asignada en la escala general del art. 21.


    ART. 4.

    La división de los delitos es aplicable a los cuasidelitos, que se califican y penan en los casos especiales que determina este Código.


    ART. 5.

    La ley penal chilena es obligatoria para todos los habitantes de la República, inclusos los extranjeros. Los delitos cometidos dentro del mar territorial o adyacente quedan sometidos a las prescripciones de este Código.


    ART. 6.

    Los crímenes o simples delitos perpetrados fuera del territorio de la República por chilenos o por extranjeros, no serán castigados en Chile sino en los casos determinados por la ley.


    ART. 7.

    Son punibles, no sólo el crimen o simple delito consumado, sino el frustrado y la tentativa.
    Hay crimen o simple delito frustrado cuando el delincuente pone de su parte todo lo necesario para que el crimen o simple delito se consume y esto no se verifica por causas independientes de su voluntad.
    Hay tentativa cuando el culpable da principio a la ejecución del crimen o simple delito por hechos directos, pero faltan uno o más para su complemento.



    ART. 8.

    La conspiración y proposición para cometer un crimen o un simple delito, sólo son punibles en los casos en que la ley las pena especialmente.
    La conspiración existe cuando dos o más personas se conciertan para la ejecución del crimen o simple delito.
    La proposición se verifica cuando el que ha resuelto cometer un crimen o un simple delito, propone su ejecución a otra u otras personas.
    Exime de toda pena por la conspiración o proposición para cometer un crimen o un simple delito, el desistimiento de la ejecución de éstos antes de principiar a ponerlos por obra y de iniciarse procedimiento judicial contra el culpable, con tal que denuncie a la autoridad pública el plan y sus circunstancias.




    ART. 9.

    Las faltas sólo se castigan cuando han sido consumadas.



    § II.

    De las circunstancias que eximen de responsabilidad criminal.


    ART. 10.

    Están exentos de responsabilidad criminal:

    1.° El loco o demente, a no ser que haya obrado en un intervalo lúcido, y el que, por cualquier causa independiente de su voluntad, se halla privado totalmente de razón.
    Inciso Derogado.
    Inciso Derogado.
    2.º El menor de dieciocho años. La responsabilidad de los menores de dieciocho años y mayores de catorce se regulará por lo dispuesto en la ley de responsabilidad penal juvenil.
    3.° Derogado.
    4.° El que obra en defensa de su persona o derechos, siempre que concurran las circunstancias siguientes:
    Primera.-Agresión Ilegítima.
    Segunda.- Necesidad racional del medio empleado para impedirla o repelerla.
    Tercera.-Falta de provocación suficiente por parte del que se defiende.
    Inciso Derogado.
    5.° El que obra en defensa de la persona o derechos de su cónyuge, de su conviviente civil, de sus parientes consanguíneos en toda la línea recta y en la colateral hasta el cuarto grado, de sus afines en toda la línea recta y en la colateral hasta el segundo grado, de sus padres o hijos, siempre que concurran la primera y segunda circunstancias prescritas en el número anterior, y la de que, en caso de haber precedido provocación de parte del acometido, no tuviere participación en ella el defensor.
    6.° El que obra en defensa de la persona y derechos de un extraño, siempre que concurran las circunstancias expresadas en el número anterior y la de que el defensor no sea impulsado por venganza, resentimiento u otro motivo ilegítimo.
    Se presumirá legalmente que concurren las circunstancias previstas en este número y en los números 4° y 5° precedentes, cualquiera que sea el daño que se ocasione al agresor, respecto de aquel que rechaza el escalamiento en los términos indicados en el número 1° del artículo 440 de este Código, en una casa, departamento u oficina habitados, o en sus dependencias o, si es de noche, en un local comercial o industrial y del que impida o trate de impedir la consumación de los delitos señalados en los artículos 141, 142, 361, 362, 365 bis, 390, 391, 433 y 436 de este Código.
    7.°  El que para evitar un mal ejecuta un hecho, que produzca daño en la propiedad ajena, siempre que concurran las circunstancias siguientes:
    Primera.-Realidad o peligro inminente del mal que se trata de evitar.
    Segunda.-Que sea mayor que el causado para evitarlo.
    Tercera.-Que no haya otro medio practicable y menos perjudicial para impedirlo.
    8.°  El que con ocasión de ejecutar un acto lícito, con la debida diligencia, causa un mal por mero accidente.
    9.°  El que obra violentado por una fuerza irresistible o impulsado por un miedo insuperable.
    10.° El que obra en cumplimiento de un deber o en el ejercicio legítimo de un derecho, autoridad, oficio o cargo.
    11.° El que obra para evitar un mal grave para su persona o derecho o los de un tercero, siempre que concurran las circunstancias siguientes:

    1ª. Actualidad o inminencia del mal que se trata de evitar.
    2ª. Que no exista otro medio practicable y menos perjudicial para evitarlo.
    3ª. Que el mal causado no sea sustancialmente superior al que se evita.
    4ª. Que el sacrificio del bien amenazado por el mal no pueda ser razonablemente exigido al que lo aparta de sí o, en su caso, a aquel de quien se lo aparta siempre que ello estuviese o pudiese estar en conocimiento del que actúa.
    12.° El que incurre en alguna omisión, hallándose impedido por causa legítima o insuperable.
    13.° El que cometiere un cuasidelito, salvo en los casos expresamente penados por la ley.





    § III.

    De las circunstancias que atenúan la responsabilidad criminal.


    ART. 11.

    Son circunstancias atenuantes:

    1.° Las expresadas en el artículo anterior, cuando no concurren todos los requisitos necesarios para eximir de responsabilidad en sus respectivos casos.
    2.° Derogado.
    3.° La de haber precedido inmediatamente de parte del ofendido, provocación o amenaza proporcionada al delito.
    4.° La de haberse ejecutado el hecho en vindicación próxima de una ofensa grave causada al autor, a su cónyuge, o su conviviente, a sus parientes legítimos por consanguinidad o afinidad en toda la línea recta y en la colateral hasta el segundo grado inclusive, a sus padres o hijos naturales o ilegítimos reconocidos.
    5.°  La de obrar por estímulos tan poderosos que naturalmente hayan producido arrebato y obcecación.
    6.°  Si la conducta anterior del delincuente ha sido irreprochable.
    7.°  Si ha procurado con celo reparar el mal causado o impedir sus ulteriores perniciosas consecuencias.
    8.°  Si pudiendo eludir la acción de la justicia por medio de la fuga u ocultándose, se ha denunciado y confesado el delito.
    9ª. Si se ha colaborado sustancialmente al esclarecimiento de los hechos.
    10.° El haber obrado por celo de la justicia.




    § IV.

    De las circunstancias que agravan la responsabilidad criminal.


    ART. 12.

    Son circunstancias agravantes:
    1.° Cometer el delito contra las personas con alevosía, entendiéndose que la hay cuando se obra a traición o sobre seguro.
    2.° Cometerlo mediante precio, recompensa o promesa.
    3.° Ejecutar el delito por medio de inundación, incendio, veneno u otro artificio que pueda ocasionar grandes estragos o dañar a otras personas.
    4.° Aumentar deliberadamente el mal del delito causando otros males innecesarios para su ejecución.
    5.° En los delitos contra las personas, obrar con premeditación conocida o emplear astucia, fraude o disfraz.
    6.° Abusar el delincuente de la superioridad de su sexo o de sus fuerzas, en términos que el ofendido no pudiera defenderse con probabilidades de repeler la ofensa.
    7.° Cometer el delito con abuso de confianza.
    8.° Prevalerse del carácter público que tenga el culpable.
    9.° Emplear medios o hacer que concurran circunstancias que añadan la ignominia a los efectos propios del hecho.
    10.° Cometer el delito con ocasión de incendio, naufragio, sedición, tumulto o conmoción popular u otra calamidad o desgracia.
    11.° Ejecutarlo con auxilio de gente armada o de personas que aseguren o proporcionen la impunidad.
    12.° Ejecutarlo de noche o en despoblado.
    El tribunal tomará o no en consideración esta circunstancia, según la naturaleza y accidentes del delito.
    13.° Ejecutarlo en desprecio o con ofensa de la autoridad pública o en el lugar en que se halle ejerciendo sus funciones.
    14.° Cometer el delito mientras cumple una condena o después de haberla quebrantado y dentro del plazo en que puede ser castigado por el quebrantamiento.
    15.° Haber sido condenado el culpable anteriormente por delitos a que la ley señale igual o mayor pena.
    16 ª Haber sido condenado el culpable anteriormente por delito de la misma especie.
    17.° Cometer el delito en lugar destinado al ejercicio de un culto permitido en la República.
    18.° Ejecutar el hecho con ofensa o desprecio del respeto que por la dignidad, autoridad, edad o sexo mereciere el ofendido, o en su morada, cuando él no haya provocado el suceso.
    19.° Ejecutarlo por medio de fractura o escalamiento de lugar cerrado.
    20.° Ejecutarlo portando armas de aquellas referidas en el artículo 132.
    21ª. Cometer el delito o participar en él motivado por la ideología, opinión política, religión o creencias de la víctima; la nación, raza, etnia o grupo social a que pertenezca; su sexo, orientación sexual, identidad de género, edad, filiación, apariencia personal o la enfermedad o discapacidad que padezca.




      § V.

De las circunstancias que atenúan o agravan la responsabilidad criminal, según la naturaleza y accidentes del delito.



    ART. 13.

    Es circunstancia atenuante o agravante, según la naturaleza y accidentes del delito.
    Ser el agraviado cónyuge o conviviente civil, pariente por consanguinidad o afinidad en toda la línea recta y en la colateral hasta el segundo grado, padre o hijo del ofensor.



    TÍTULO SEGUNDO.

    DE LAS PERSONAS RESPONSABLES DE LOS DELITOS.


    ART. 14.

    Son responsables criminalmente de los delitos:
    1.° Los autores.
    2.° Los cómplices.
    3.° Los encubridores.


    ART. 15.

    Se consideran autores:

    1.° Los que toman parte en la ejecución del hecho, sea de una manera inmediata y directa; sea impidiendo o procurando impedir que se evite.
    2.° Los que fuerzan o inducen directamente a otro a ejecutarlo.
    3.° Los que, concertados para su ejecución, facilitan los medios con que se lleva a efecto el hecho o lo presencian sin tomar parte inmediata en él.


    ART. 16.

    Son cómplices los que, no hallándose comprendidos en el artículo anterior, cooperan a la ejecución del hecho por actos anteriores o simultáneos.


    ART. 17.

    Son encubridores los que con conocimiento de la perpetración de un crimen o de un simple delito o de los actos ejecutados para llevarlo a cabo, sin haber tenido participación en él como autores ni como cómplices, intervienen, con posterioridad a su ejecución, de alguno de los modos siguientes:

    1.° Aprovechándose por sí mismos o facilitando a los delincuentes medios para que se aprovechen de los efectos del crimen o simple delito.
    2.° Ocultando o inutilizando el cuerpo, los efectos o instrumentos del crimen o simple delito para impedir su descubrimiento.
    3.° Albergando, ocultando o proporcionando la fuga del culpable.
    4.° Acogiendo, receptando o protegiendo habitualmente a los malhechores, sabiendo que lo son, aun sin conocimiento de los crímenes o simples delitos determinados que hayan cometido, o facilitándoles los medios de reunirse u ocultar sus armas o efectos, o suministrándoles auxilios o noticias para que se guarden, precavan o salven.
    Están exentos de las penas impuestas a los encubridores los que lo sean de su cónyuge, de su conviviente civil, o de sus parientes por consanguinidad o afinidad en toda la línea recta y en la colateral hasta el segundo grado, de sus padres o hijos, con la sola excepción de los que se hallaren comprendidos en el número 1° de este artículo.


    TÍTULO TERCERO.

    DE LAS PENAS.



    § I.

    De las penas en general.


    ART. 18.

    Ningún delito se castigará con otra pena que la que le señale una ley promulgada con anterioridad a su perpetración.
    Si después de cometido el delito y antes de que se pronuncie sentencia de término, se promulgare otra ley que exima tal hecho de toda pena o le aplique una menos rigorosa, deberá arreglarse a ella su juzgamiento.
    Si la ley que exima el hecho de toda pena o le aplique una menos rigurosa se promulgare después de ejecutoriada la sentencia, sea que se haya cumplido o no la condena impuesta, el tribunal que hubiere pronunciado dicha sentencia, en primera o única instancia, deberá modificarla de oficio o a petición de parte.
    En ningún caso la aplicación de este artículo modificará las consecuencias de la sentencia primitiva en lo que diga relación con las indemnizaciones pagadas o cumplidas o las inhabilidades.


    ART. 19.

    El perdón de la parte ofendida no extingue la acción penal, salvo respecto de los delitos que no pueden ser perseguidos sin previa denuncia o consentimiento del agraviado.


    ART. 20.

    No se reputan penas, la restricción o privación de libertad de los detenidos o sometidos a prisión preventiva u otras medidas cautelares personales, la separación de los empleos públicos acordada por las autoridades en uso de sus atribuciones o por el tribunal durante el proceso o para instruirlo, ni las multas y demás correcciones que los superiores impongan a sus subordinados y administrados en uso de su jurisdicción disciplinal o atribuciones gubernativas.

    § II.

    De la clasificación de las penas.


    ART. 21.

    Las penas que pueden imponerse con arreglo a este Código y sus diferentes clases, son las que comprende la siguiente:

    ESCALA GENERAL.

    PENAS DE CRÍMENES.
    Presidio perpetuo calificado.
    Presidio perpetuo.
    Reclusión perpetua.
    Presidio mayor.
    Reclusión mayor.
    Relegación perpetua.
    Confinamiento mayor.
    Extrañamiento mayor.
    Relegación mayor.
    Inhabilitación absoluta perpetua para cargos y oficios públicos, derechos políticos y profesiones titulares.
    Inhabilitación absoluta perpetua para cargos, empleos, oficios o profesiones ejercidos en ámbitos educacionales o que involucren una relación directa y habitual con personas menores de edad.
    Inhabilitación absoluta perpetua para cargos, empleos, oficios o profesiones ejercidos en ámbitos educacionales, de la salud o que involucren una relación directa y habitual con menores de dieciocho años de edad, adultos mayores o personas en situación de discapacidad.
    Inhabilitación absoluta perpetua para ejercer cargos, empleos, oficios o profesiones en empresas que contraten con órganos o empresas del Estado o con empresas o asociaciones en que éste tenga una participación mayoritaria; o en empresas que participen en concesiones otorgadas por el Estado o cuyo objeto sea la provisión de servicios de utilidad pública.
    Inhabilitación especial perpetua para algún cargo u oficio público o profesión titular.
    Inhabilitación absoluta temporal para ejercer cargos, empleos, oficios o profesiones en empresas que contraten con órganos o empresas del Estado o con empresas o asociaciones en que éste tenga una participación mayoritaria; o en empresas que participen en concesiones otorgadas por el Estado o cuyo objeto sea la provisión de servicios de utilidad pública.
    Inhabilitación absoluta temporal para cargos, empleos, oficios o profesiones ejercidos en ámbitos educacionales o que involucren una relación directa y habitual con personas menores de edad.
    Inhabilitación absoluta temporal para cargos, empleos, oficios o profesiones ejercidos en ámbitos educacionales, de la salud o que involucren una relación directa y habitual con menores de dieciocho años de edad, adultos mayores o personas en situación de discapacidad.
    Inhabilitación absoluta temporal para cargos y oficios públicos y profesiones titulares.
    Inhabilitación especial temporal para algún cargo u oficio público o profesión titular.

    PENAS DE SIMPLES DELITOS.

    Presidio menor.
    Reclusión menor.
    Confinamiento menor.
    Extrañamiento menor.
    Relegación menor.
    Destierro.
    Inhabilitación absoluta temporal para cargos, empleos, oficios o profesiones ejercidos en ámbitos educacionales o que involucren una relación directa y habitual con personas menores de edad.
    Inhabilitación absoluta temporal para cargos, empleos, oficios o profesiones ejercidos en ámbitos educacionales, de la salud o que involucren una relación directa y habitual con menores de dieciocho años de edad, adultos mayores o personas en situación de discapacidad.
    Inhabilitación absoluta temporal para ejercer cargos, empleos, oficios o profesiones en empresas que contraten con órganos o empresas del Estado o con empresas o asociaciones en que éste tenga una participación mayoritaria; o en empresas que participen en concesiones otorgadas por el Estado o cuyo objeto sea la provisión de servicios de utilidad pública.
    Inhabilitación especial temporal para emitir licencias médicas.
    Suspensión de cargo u oficio público o profesión titular.
    Inhabilidad perpetua para conducir vehículos a tracción mecánica o animal.
    Suspensión para conducir vehículos a tracción mecánica o animal.
    Inhabilidad absoluta perpetua para la tenencia de animales.

    PENAS DE LAS FALTAS.

    Prisión.
    Inhabilidad perpetua para conducir vehículos a tracción mecánica o animal.
    Suspensión para conducir vehículos a tracción mecánica o animal.

    PENAS COMUNES A LAS TRES CLASES ANTERIORES.

    Multa.
    Pérdida o comiso de los instrumentos o efectos del delito.

    PENAS ACCESORIAS DE LOS CRÍMENES Y SIMPLES DELITOS.

    ELIMINADA.
    Incomunicación con personas extrañas al establecimiento penal, en conformidad al Reglamento carcelario.

    Penas sustitutivas por vía de conversión de la multa
    Prestación de servicios en beneficio de la comunidad.








    ART. 22.

    Son penas accesorias las de suspensión e inhabilitación para cargos y oficios públicos, derechos políticos y profesiones titulares en los casos en que, no imponiéndolas especialmente la ley, ordena que otras penas las lleven consigo.



    ART. 23.

    La caución y la sujeción a la vigilancia de la autoridad podrán imponerse como penas accesorias o como medidas preventivas, en los casos especiales que determinen este Código y el de Procedimientos.


    ART. 24.

    Toda sentencia condenatoria en materia criminal lleva envuelta la obligación de pagar las costas, daños y perjuicios por parte de los autores, cómplices, encubridores y demás personas legalmente responsables.


   
    § III.

   
    De los límites, naturaleza y efectos de las penas.

    ART. 25.

    Las penas temporales mayores duran de cinco años y un día a veinte años, y las temporales menores de sesenta y un días a cinco años.
    Las de inhabilitación absoluta y especial temporales para cargos y oficios públicos y profesiones titulares duran de tres años y un día a diez años.
    La suspensión de cargo u oficio público o profesión titular, dura de sesenta y un días a tres años.
    Las penas de destierro y de sujeción a la vigilancia de la autoridad, de sesenta y un días a cinco años.
    La prisión dura de uno a sesenta días.
    La cuantía de la multa, tratándose de crímenes, no podrá exceder de treinta unidades tributarias mensuales; en los simples delitos, de veinte unidades tributarias mensuales, y en las faltas, de cuatro unidades tributarias mensuales; todo ello, sin perjuicio de que en determinadas infracciones, atendida su gravedad, se contemplen multas de cuantía superior.
    La expresión "unidad tributaria mensual" en cualquiera disposición de este Código, del Código de Procedimiento Penal y demás leyes penales especiales significa una unidad tributaria mensual vigente a la fecha de comisión del delito, y, tratándose de multas, ellas se deberán pagar en pesos, en el valor equivalente que tenga la unidad tributaria mensual al momento de su pago.
    Cuando la ley impone multas cuyo cómputo debe hacerse en relación a cantidades indeterminadas, nunca podrán aquéllas exceder de treinta unidades tributarias mensuales.
    En cuanto a la cuantía de la caución, se observarán las reglas establecidas para la multa, doblando las cantidades respectivamente, y su duración no podrá exceder del tiempo de la pena u obligación cuyo cumplimiento asegura, o de cinco años en los demás casos.
    INCISO SUPRIMIDO.




    ART. 26.

    La duración de las penas temporales empezará a contarse desde el día de la aprehensión del imputado.

    PENAS QUE LLEVAN CONSIGO OTRAS ACCESORIAS.


    ART. 27.

    Las penas de presidio, reclusión y relegación perpetuos, llevan consigo la de inhabilitación absoluta perpetua para cargos y oficios públicos y derechos políticos por el tiempo de la vida de los penados y la de sujeción a la vigilancia de la autoridad por el máximum que establece este Código.



    ART. 28.

    Las penas de presidio, reclusión, confinamiento, extrañamiento y relegación mayores, llevan consigo la de inhabilitación absoluta perpetua para cargos y oficios públicos y derechos políticos y la de inhabilitación absoluta para profesiones titulares mientras dure la condena.


    ART. 29.

    Las penas de presidio, reclusión, confinamiento, extrañamiento y relegación menores en sus grados máximos, llevan consigo la de inhabilitación absoluta perpetua para derechos políticos y la de inhabilitación absoluta para cargos y oficios públicos durante el tiempo de la condena.


    ART. 30.

    Las penas de presidio, reclusión, confinamiento, extrañamiento y relegación menores en sus grados medios y mínimos, y las de destierro y prisión, llevan consigo la de suspensión de cargo u oficio público durante el tiempo de la condena.



    ART. 31.

    Toda pena que se imponga por un crimen o un simple delito, lleva consigo la pérdida de los efectos que de él provengan y de los instrumentos con que se ejecutó, a menos que pertenezcan a un tercero no responsable del crimen o simple delito.


    NATURALEZA Y EFECTOS DE ALGUNAS PENAS.


    ART. 32.

    La pena de presidio sujeta al condenado a los trabajos prescritos por los reglamentos del respectivo establecimiento penal. Las de reclusión y prisión no le imponen trabajo alguno.


    ART. 32. BIS

    La imposición del presidio perpetuo calificado importa la privación de libertad del condenado de por vida, bajo un régimen especial de cumplimiento que se rige por las siguientes reglas:

    1.ª No se podrá conceder la libertad condicional sino una vez transcurridos cuarenta años de privación de libertad efectiva, debiendo en todo caso darse cumplimiento a las demás normas y requisitos que regulen su otorgamiento y revocación;

    2.ª El condenado no podrá ser favorecido con ninguno de los beneficios que contemple el reglamento de establecimientos penitenciarios, o cualquier otro cuerpo legal o reglamentario, que importe su puesta en libertad, aun en forma transitoria. Sin perjuicio de ello, podrá autorizarse su salida, con las medidas de seguridad que se requieran, cuando su cónyuge, su conviviente civil, o alguno de sus padres o hijos se encontrare en inminente riesgo de muerte o hubiere fallecido;

    3.ª No se favorecerá al condenado por las leyes que concedan amnistía ni indultos generales, salvo que se le hagan expresamente aplicables. Asimismo, sólo procederá a su respecto el indulto particular por razones de Estado o por el padecimiento de un estado de salud grave e irrecuperable, debidamente acreditado, que importe inminente riesgo de muerte o inutilidad física de tal magnitud que le impida valerse por sí mismo. En todo caso el beneficio del indulto deberá ser concedido de conformidad a las normas legales que lo regulen.




    ART. 33.

    Confinamiento es la expulsión del condenado del territorio de la República con residencia forzosa en un lugar determinado.





    ART. 34.

    Extrañamiento es la expulsión del condenado del territorio de la República al lugar de su elección.





    ART. 35.

    Relegación es la traslación del condenado a un punto habitado del territorio de la República con prohibición de salir de él, pero permaneciendo en libertad.



    ART. 36.

    Destierro es la expulsión del condenado de algún punto de la República.




    ART. 37.

    Para los efectos legales se reputan aflictivas todas las penas de crímenes y, respecto de las de simples delitos, las de presidio, reclusión, confinamiento, extrañamiento y relegación menores en sus grados máximos.


    ART. 38.

    La pena de inhabilitación absoluta perpetua para cargos y oficios públicos, derechos políticos y profesiones titulares, y la de inhabilitación absoluta temporal para cargos y oficios públicos Y profesiones titulares, producen:
    1.° La privación de todos los honores, cargos, empleos y oficios públicos y profesiones titulares de que estuviere en posesión el penado, aun cuando sean de elección popular.
    2.° La privación de todos los derechos políticos activos y pasivos y la incapacidad perpetua para obtenerlos.
    3.° La incapacidad para obtener los honores, cargos, empleos, oficios y profesiones mencionados, perpetuamente si la inhabilitación es perpetua y durante el tiempo de la condena si es temporal.
    4.° Derogado.




    ART. 39.

    Las penas de inhabilitación especial perpetua y temporal para algún cargo u oficio público o profesión titular, producen:
    1.° La privación del cargo, empleo, oficio o profesión sobre que recaen, y la de los honores anexos a él, perpetuamente si la inhabilitación es perpetua, y por el tiempo de la condena si es temporal.
    2.° La incapacidad para obtener dicho cargo, empleo, oficio o profesión u otros en la misma carrera, perpetuamente cuando la inhabilitación es perpetua, y por el tiempo de la condena cuando es temporal.

    ART. 39 bis.

    Las penas de inhabilitación absoluta perpetua o temporal para cargos, empleos, oficios o profesiones ejercidos en ámbitos educacionales o que involucren una relación directa y habitual con personas menores de edad, prevista en el artículo 372 de este Código, produce:

    1º La privación de todos los cargos, empleos, oficios y profesiones ejercidos en ámbitos educacionales o que involucren una relación directa y habitual con personas menores de edad que tenga el condenado.
    2º La incapacidad para obtener los cargos, empleos, oficios y profesiones mencionados, perpetuamente cuando la inhabilitación es perpetua, y si la inhabilitación es temporal, la incapacidad para obtenerlos, antes de transcurrido el tiempo de la condena de inhabilitación, contado desde que se hubiere dado cumplimiento a la pena principal, obtenido libertad condicional en la misma, o iniciada la ejecución de alguna de las penas de la ley Nº 18.216 como sustitutiva de la pena principal.
    La pena de inhabilitación absoluta temporal de que trata este artículo tiene una extensión de tres años y un día a diez años y es divisible en la misma forma que las penas de inhabilitación absoluta y especial temporales.


    ART. 39 ter.

    La pena de inhabilitación absoluta perpetua o temporal para cargos, empleos, oficios o profesiones ejercidos en ámbitos educacionales, de la salud o que involucren una relación directa y habitual con menores de dieciocho años de edad, adultos mayores o personas en situación de discapacidad, prevista en el artículo 403 quáter de este código, produce:

    1º. La privación de todos los cargos, empleos, oficios y profesiones que tenga el condenado, ejercidos en ámbitos educacionales, de la salud o que involucren una relación directa y habitual con las personas mencionadas en el inciso primero de este artículo.
    2º. La incapacidad para obtener los cargos, empleos, oficios y profesiones mencionados, perpetuamente cuando la inhabilitación es perpetua, y por el tiempo de la condena cuando es temporal.

    La pena de inhabilitación absoluta temporal de que trata este artículo tiene una extensión de tres años y un día a diez años y es divisible en la misma forma que las penas de inhabilitación absoluta y especial temporales.




    ART. 39 quáter.- La pena de inhabilitación absoluta perpetua o temporal para ejercer cargos, empleos, oficios o profesiones en empresas que contraten con órganos o empresas del Estado o con empresas o asociaciones en que éste tenga una participación mayoritaria; o en empresas que participen en concesiones otorgadas por el Estado o cuyo objeto sea la provisión de servicios de utilidad pública, prevista en el artículo 251 quáter de este Código, produce:
     
    1º. La privación de todos los cargos, empleos, oficios y profesiones ejercidos en empresas que contraten con órganos o empresas del Estado o con empresas o asociaciones en que éste tenga una participación mayoritaria; o en empresas que participen en concesiones otorgadas por el Estado o cuyo objeto sea la provisión de servicios de utilidad pública.
    2º. La incapacidad para obtener los cargos, empleos, oficios y profesiones mencionados, perpetuamente cuando la inhabilitación es perpetua, y por el tiempo de la condena cuando es temporal.
     
    La pena de inhabilitación absoluta temporal de que trata este artículo tiene una extensión de tres años y un día a diez años y es divisible en la misma forma que las penas de inhabilitación absoluta y especial temporales.
    En este caso, ejecutoriada que sea la sentencia definitiva, el tribunal la comunicará a la Dirección de Compras y Contratación Pública. Dicha Dirección mantendrá un registro público actualizado de las personas naturales a las que se les haya impuesto esta pena.




    ART. 40.

    La suspensión de cargo y oficio público y profesión titular, inhabilita para su ejercicio durante el tiempo de la condena.
    La suspensión decretada durante el juicio, trae como consecuencia inmediata la privación de la mitad del sueldo al imputado, la cual sólo se le devolverá en el caso de pronunciarse sentencia absolutoria.
    La suspensión decretada por vía de pena, priva de todo sueldo al suspenso mientras ella dure.

    ART. 41.

    Cuando las penas de inhabilitación y suspensión recaigan en persona eclesiástica, sus efectos no se extenderán a los cargos, derechos y honores que tenga por la Iglesia. A los eclesiásticos incursos en tales penas y por todo el tiempo de su duración, no se les reconocerá en la República la jurisdicción eclesiástica y la cura de almas, ni podrán percibir rentas del tesoro nacional, salvo la congrua que fijará el tribunal.
    Esta disposición no comprende a los obispos en lo concerniente al ejercicio de la jurisdicción ordinaria que les corresponde.


    ART. 42.

    Los derechos políticos activos y pasivos a que se refieren los artículos anteriores, son: la capacidad para ser ciudadano elector, la capacidad para obtener cargos de elección popular y la capacidad para ser jurado.   

    El que ha sido privado de ellos sólo puede ser rehabilitado en su ejercicio en la forma prescrita por la Constitución.




    ART. 43.

    Cuando la inhabilitación para cargos y oficios públicos y profesiones titulares es pena accesoria, no la comprende el indulto de la pena principal, a menos que expresamente se haga extensivo a ella.


    ART. 44.

    El indulto de la pena de inhabilitación perpetua o temporal para cargos y oficios públicos y profesiones titulares, repone al penado en el ejercicio de estas últimas, pero no en los honores, cargos, empleos u oficios de que se le hubiere privado. El mismo efecto produce el cumplimiento de la condena a inhabilitación temporal.



    ART. 45.

    La sujeción a la vigilancia de la autoridad da al juez de la causa el derecho de determinar ciertos lugares en los cuales le será prohibido al penado presentarse después de haber cumplido su condena y de imponer a éste todas o algunas de las siguientes obligaciones:
    1.° La de declarar antes de ser puesto en libertad, el lugar en que se propone fijar su residencia.
    2.° La de recibir una boleta de viaje en que se le determine el itinerario que debe seguir, del cual no podrá apartarse, y la duración de su permanencia en cada lugar del tránsito.
    3.° La de presentarse dentro de las veinticuatro horas siguientes a su llegada, ante el funcionario designado en la boleta de viaje.
    4.° La de no poder cambiar de residencia sin haber dado aviso de ello, con tres días de anticipación, al mismo funcionario, quien le entregará la boleta de viaje primitiva visada para que se traslade a su nueva residencia.
    5.a La de adoptar oficio, arte, industria o profesión, si no tuviere medios propios y conocidos de subsistencia.


    ART. 46.

    La pena de caución produce en el penado la obligación de presentar un fiador abonado que responda o bien de que aquél no ejecutará el mal que se trata de precaver, o de que cumplirá su condena; obligándose a satisfacer, si causare el mal o quebrantare la condena, la cantidad que haya fijado el tribunal.
    Si el penado no presentare fiador, sufrirá una reclusión equivalente a la cuantía de la fianza, computándose un día por cada quinto de unidad tributaria mensual; pero sin poder en ningún caso exceder de seis meses.





    ART. 47.

    En todos los casos en que se imponga el pago de costas se entenderá comprender tanto las procesales como las personales y además los gastos ocasionados por el juicio y que no se incluyen en las costas. Estos gastos se fijarán por el tribunal, previa audiencia de las partes.



    ART. 48.

    Si los bienes del culpable no fueren bastantes para cubrir las responsabilidades pecuniarias, se satisfarán éstas en el orden siguiente:
    1.° Las costas procesales y personales.
    2.° El resarcimiento de los gastos ocasionados por el juicio.
    3.° La reparación del daño causado e indemnización de perjuicios.
    4.° La multa.
    En caso de un procedimiento concursal, estos créditos se graduarán, considerándose como uno solo, entre los que no gozan de preferencia.



    ART. 49.

    Si el sentenciado no tuviere bienes para satisfacer la multa podrá el tribunal imponer, por vía de sustitución, la pena de prestación de servicios en beneficio de la comunidad.
    Para proceder a esta sustitución se requerirá del acuerdo del condenado. En caso contrario, el tribunal impondrá, por vía de sustitución y apremio de la multa, la pena de reclusión, regulándose un día por cada tercio de unidad tributaria mensual, sin que ella pueda nunca exceder de seis meses.
    No se aplicará la pena sustitutiva señalada en el inciso primero ni se hará efectivo el apremio indicado en el inciso segundo, cuando, de los antecedentes expuestos por el condenado, apareciere la imposibilidad de cumplir la pena.
    Queda también exento de este apremio el condenado a reclusion menor en su grado máximo o a otra pena mas grave que deba cumplir efectivamente.



    Art. 49 bis.

    La pena de prestación de servicios en beneficio de la comunidad consiste en la realización de actividades no remuneradas a favor de ésta o en beneficio de personas en situación de precariedad, coordinadas por un delegado de Gendarmería de Chile.
    El trabajo en beneficio de la comunidad será facilitado por Gendarmería, pudiendo establecer los convenios que estime pertinentes para tal fin con organismos públicos y privados sin fines de lucro.
    Gendarmería de Chile y sus delegados, y los organismos públicos y privados que en virtud de los convenios a que se refiere el inciso anterior intervengan en la ejecución de esta sanción, deberán velar por que no se atente contra la dignidad del penado en la ejecución de estos servicios.


    Art. 49 ter.

    La pena de prestación de servicios en beneficio de la comunidad se regulará en ocho horas por cada tercio de unidad tributaria mensual, sin perjuicio de la conversión establecida en leyes especiales.
    Su duración diaria no podrá exceder de ocho horas.
    En cualquier momento el condenado podrá solicitar poner término a la prestación de servicios en beneficio de la comunidad previo pago de la multa, a la que se deberán abonar las horas trabajadas.



    Art. 49 quáter.

    En caso de decretarse la sanción de prestación de servicios en beneficio de la comunidad, el delegado de Gendarmería de Chile responsable de gestionar el cumplimiento informará al tribunal que dictó la sentencia, quien a su vez notificará al Ministerio Público, al defensor y al condenado, el tipo de servicio, el lugar donde deba realizarse y el calendario de su ejecución, dentro de los treinta días siguientes a la fecha en que la condena se encontrare firme o ejecutoriada.


    Art. 49 quinquies.

    En caso de incumplimiento de la pena de servicios en beneficio de la comunidad, el delegado deberá informar al tribunal que haya impuesto la sanción.
    El tribunal citará a una audiencia para resolver la mantención o la revocación de la pena.



    Art. 49 sexies.

    El juez podrá revocar la pena de servicios en beneficio de la comunidad cuando el condenado:
    a) No se presentare, injustificadamente, ante Gendarmería de Chile a cumplir la pena en el plazo que determine el juez, el que no podrá ser menor a tres ni superior a siete días;
    b) Se ausentare del trabajo durante al menos dos jornadas laborales. Si el penado faltare al trabajo por causa justificada no se entenderá dicha ausencia como abandono de la actividad;
    c) Su rendimiento en la ejecución de los servicios fuere sensiblemente inferior al mínimo exigible, a pesar de los requerimientos del responsable del centro de trabajo, o
    d) Se opusiere o incumpliere de forma reiterada y manifiesta las instrucciones que se le dieren por el responsable del centro de trabajo.
    En caso de revocar la pena de servicios en beneficio de la comunidad, el tribunal impondrá al condenado, por vía de sustitución y apremio de la multa originalmente impuesta, la pena de reclusión, regulándose un día por cada tercio de unidad tributaria mensual, sin que ella pueda nunca exceder de seis meses.
    Habiéndose decretado la revocación se abonará al tiempo de reclusión un día por cada ocho horas efectivamente trabajadas en beneficio de la comunidad.
    Si el tribunal no revocare la pena de servicios en beneficio de la comunidad podrá ordenar que el cumplimiento de la misma se lleve a cabo en un lugar distinto a aquel en el cual originalmente se estaba ejecutando; todo lo anterior sin perjuicio de la facultad prevista en el inciso tercero del artículo 49.



    § IV.

    De la aplicación de las penas.


    ART. 50.

    A los autores de delito se impondrá la pena que para éste se hallare señalada por la ley.
    Siempre que la ley designe la pena de un delito, se entiende que la impone al delito consumado.


    ART. 51.

    A los autores de crimen o simple delito frustrado y a los cómplices de crimen o simple delito consumado, se impondrá la pena inmediatamente inferior en grado a la señalada por la ley para el crimen o simple delito.


    ART. 52.

    A los autores de tentativa de crimen o simple delito, a los cómplices de crimen o simple delito frustrado y a los encubridores de crimen o simple delito consumado, se impondrá la pena inferior en dos grados a la que señala la ley para el crimen o simple delito.

    Exceptúanse de esta regla los encubridores comprendidos en el núm. 3.° del art. 17, en quienes concurra la circunstancia primera del mismo número, a los cuales se impondrá la pena de inhabilitación especial perpetua, si el delincuente encubierto fuere condenado por crimen y la de inhabilitación especial temporal en cualquiera de sus grados, si lo fuere por simple delito.

    También se exceptúan los encubridores comprendidos en el núm. 4.° del mismo art. 17, a quienes se aplicará la pena de presidio menor en cualquiera de sus grados.


    ART. 53.

    A los cómplices de tentativa de crimen o simple delito y a los encubridores de crimen o simple delito frustrado, se impondrá la pena inferior en tres grados a la que señala la ley para el crimen o simple delito.


    ART. 54.

    A los encubridores de tentativa de crimen o simple delito, se impondrá la pena inferior en cuatro grados a la señalada para el crimen o simple delito.


    ART. 55.

    Las disposiciones generales contenidas en los cuatro artículos precedentes no tienen lugar en los casos en que el delito frustrado, la tentativa, la complicidad o el encubrimiento se hallan especialmente penados por la ley.


    ART. 56.

    Las penas divisibles constan de tres grados, mínimo, medio y máximo, cuya extensión se determina en la siguiente:

    TABLA DEMOSTRATIVA

  



    ART. 57.

    Cada grado de una pena divisible constituye pena distinta.


    ART. 58.

    En los casos en que la ley señala una pena compuesta de dos o más distintas, cada una de éstas forma un grado de penalidad, la más leve de ellas el mínimo y la más grave el máximo.



    ART. 59.

    Para determinar las penas que deben imponerse según los arts. 51, 52, 53 y 54: 1.° a los autores de crimen o simple delito frustrado; 2.° a los autores de tentativa de crimen o simple delito, cómplices de crimen o simple delito frustrado y encubridores de crimen o simple delito consumado; 3.° a los cómplices de tentativa de crimen o simple delito y encubridores de crimen o simple delito frustrado, y 4.° a los encubridores de tentativa de crimen o simple delito, el tribunal tomará por base las siguientes escalas graduales:

                    ESCALA NUMERO 1
    Grados.
    1° Presidio perpetuo calificado.
    2° Presidio o reclusión perpetuos.
    3° Presidio o reclusión mayores en sus grados máximos.
    4° Presidio o reclusión mayores en sus grados medios.
    5° Presidio o reclusión mayores en sus grados mínimos.
    6° Presidio o reclusión menores en sus grados máximos.
    7° Presidio o reclusión menores en sus grados medios.
    8° Presidio o reclusión menores en sus grados mínimos.
    9° Prisión en su grado máximo.
    10. Prisión en su grado medio.
    11. Prisión en su grado mínimo.

                    ESCALA NUMERO 2
    Grados.
    1° Relegación perpetua.
    2° Relegación mayor en su grado máximo.
    3° Relegación en su grado medio.
    4° Relegación mayor en su grado mínimo.
    5° Relegación menor en su grado máximo.
    6° Relegación menor en su grado medio.
    7° Relegación menor en su grado mínimo.
    8° Destierro en su grado máximo.
    9° Destierro en su grado medio.
    10. Destierro en su grado mínimo.

                  ESCALA NUMERO 3
    Grados.
    1° Confinamiento o extrañamiento mayores en sus grados máximos.
    2° Confinamiento o extrañamiento mayores en sus grados medios.
    3° Confinamiento o extrañamiento mayores en sus grados mínimos.
    4° Confinamiento o extrañamiento menores en sus grados máximos.
    5° Confinamiento o extrañamiento menores en sus grados medios.
    6° Confinamiento o extrañamiento menores en sus grados mínimos.
    7° Destierro en su grado máximo.
    8° Destierro en su grado medio.
    9° Destierro en su grado mínimo.

                  ESCALA NUMERO 4
    Grados.
    1° Inhabilitación absoluta perpetua.
    2° Inhabilitación absoluta temporal en su grado máximo.
    3° Inhabilitación absoluta temporal en su grado medio.
    4° Inhabilitación absoluta temporal en su grado mínimo.
    5° Suspensión en su grado máximo.
    6° Suspensión en su grado medio.
    7° Suspensión en su grado mínimo.

                    ESCALA NUMERO 5
    Grados.
    1° Inhabilitación especial perpetua.
    2° Inhabilitación especial temporal en su grado máximo.
    3° Inhabilitación especial temporal en su grado medio.
    4° Inhabilitación especial temporal en su grado mínimo.
    5° Suspensión en su grado máximo.
    6° Suspensión en su grado medio.
    7° Suspensión en su grado mínimo.


    ART. 60.

    La multa se considera como la pena inmediatamente inferior a la última en todas las escalas graduales.
    Para fijar su cuantía respectiva se adoptará la base establecida en el art. 25, y en cuanto a su aplicación a cada caso especial se observará lo que prescribe el art. 70.
    El producto de las multas, ya sea que se impongan por sentencia o que resulten de un decreto que conmuta alguna pena, ingresará en una cuenta fiscal, especial, contra la cual sólo podrá girar el Ministerio de Justicia, para algunos de los siguientes fines, y en conformidad al Reglamento que para tal efecto dictará el Presidente de la República:
    1.- Creación, instalación y mantenimiento de establecimientos penales y de reeducación de antisociales;
    2.- Creación de Tribunales e instalación, mantenimiento y desarrollo de los servicios judiciales, y
    3.- Mantenimiento de los Servicios del Patronato Nacional de Reos.
    La misma regla señalada en el inciso anterior, se aplicará respecto a las cauciones que se hagan efectivas, de los dineros que caigan en comiso y del producto de la enajenación en subasta pública de las demás especies decomisadas, la cual se deberá efectuar por la Dirección de Aprovisionamiento del Estado.
    Las disposiciones de los dos incisos anteriores no son aplicables a las multas señaladas en el artículo 483-b.
    El producto de las multas, cauciones y comisos derivados de faltas y contravenciones, se aplicará a fondos de la Municipalidad correspondiente al territorio donde se cometió el delito que se castiga.



    ART. 61.

    La designación de las penas que corresponde aplicar en los diversos casos a que se refiere el art. 59, se hará con sujeción a las siguientes reglas:

    1.° Si la pena señalada al delito es una indivisible o un solo grado de otra divisible, corresponde a los autores de crimen o simple delito frustrado y a los cómplices de crimen o simple delito consumado la inmediatamente inferior en grado.
    Para determinar las que deben aplicarse a los demás responsables relacionados en el art. 59, se bajará sucesivamente un grado en la escala correspondiente respecto de los comprendidos en cada uno de sus números, siguiendo el orden que en ese artículo se establece.
    2.° Cuando la pena que se señala al delito consta de dos o más grados, sea que los compongan dos penas indivisibles, diversos grados de penas divisibles o bien una o dos indivisibles y uno o más grados de otra divisible, a los autores de crimen o simple delito frustrado y a los cómplices de crimen o simple delito consumado corresponde la inmediatamente inferior en grado al mínimo de los designados por la ley.
    Para determinar las que deben aplicarse a los demás responsables se observará lo prescrito en la regla anterior.
    3.° Si se designan para un delito penas alternativas, sea que se hallen comprendidas en la misma escala o en dos o más distintas, no estará obligado el tribunal a imponer a todos los responsables las de la misma naturaleza.
    4.° Cuando se señalan al delito copulativamente penas comprendidas en distintas escalas o se agrega la multa a las de la misma escala, se aplicarán unas y otras, con sujeción a las reglas 1.° y 2.°, a todos los responsables; pero cuando una de dichas penas se impone al autor de crimen o simple delito por circunstancias peculiares a él que no concurren en los demás, no se hará extensiva a éstos.
    5.°  Si al poner en práctica las reglas precedentes no resultare pena que imponer por falta de grados inferiores o por no ser aplicables las de inhabilitación o suspensión, se impondrá siempre la multa.

    APLICACION PRACTICA DE LAS REGLAS ANTERIORES 




NOTA
      El Art. 1° de la ley 19450, publicada el 18.03.1996, modificada por la ley 19501, publicada el 15.05.1997, dispuso la sustitución de las escalas de multas establecidas en sueldos vitales en el Código Penal por otras expresadas en unidades tributarias mensuales o fracción de unidad tributaria mensual, de acuerdo con la tabla de conversión que establece.

    ART. 62.

    Las circunstancias atenuantes o agravantes se tomarán en consideración para disminuir o aumentar la pena en los casos y conforme a las reglas que se prescriben en los artículos siguientes.



    ART. 63.

    No producen el efecto de aumentar la pena las circunstancias agravantes que por sí mismas constituyen un delito especialmente penado por la ley, o que ésta haya expresado al describirlo y penarlo.
    Tampoco lo producen aquellas circunstancias agravantes de tal manera inherentes al delito que sin la concurrencia de ellas no puede cometerse.


    ART. 64.

    Las circunstancias atenuantes o agravantes que consistan en la disposición moral del delincuente, en sus relaciones particulares con el ofendido o en otra causa personal, servirán para atenuar o agravar la responsabilidad de sólo aquellos autores, cómplices o encubridores en quienes concurran.
    Las que consistan en la ejecución material del hecho o en los medios empleados para realizarlo, servirán para atenuar o agravar la responsabilidad únicamente de los que tuvieren conocimiento de ellas antes o en el momento de la acción o de su cooperación para el delito.




    ART. 65.

    Cuando la ley señala una sola pena indivisible, la aplicará el tribunal sin consideración a las circunstancias agravantes que concurran en el hecho. Pero si hay dos o más circunstancias atenuantes y no concurre ninguna agravante, podrá aplicar la pena inmediatamente inferior en uno o dos grados.




    ART. 66.

    Si la ley señala una pena compuesta de dos indivisibles y no acompañan al hecho circunstancias atenuantes ni agravantes, puede el tribunal imponerla en cualquiera de sus grados.
    Cuando solo concurre alguna circunstancia atenuante, debe aplicarla en su grado mínimo, y si habiendo una circunstancia agravante, no concurre ninguna atenuante, la impondrá en su grado máximo.
    Siendo dos o más las circunstancias atenuantes sin que concurra ninguna agravante, podrá imponer la pena inferior, en uno o dos grados al mínimo de los señalados por la ley, según sea el número y entidad de dichas circunstancias.
    Si concurrieren circunstancias atenuantes y agravantes, las compensará racionalmente el tribunal para la aplicación de la pena, graduando el valor de unas y otras.



    ART. 67.

    Cuando la pena señalada al delito es un grado de una divisible y no concurren circunstancias atenuantes ni agravantes en el hecho, el tribunal puede recorrer toda su extensión al aplicarla.
    Si concurre sólo una circunstancia atenuante o sólo una agravante, la aplicará en el primer caso en su mínimum y en el segundo en su máximum.
    Para determinar en tales casos el mínimum y el máximum de la pena, se divide por mitad el período de su duración: la más alta de estas partes formará el máximum y la más baja el mínimum.
    Siendo dos o más las circunstancias atenuantes y no habiendo ninguna agravante, podrá el tribunal imponer la inferior en uno o dos grados, según sea el número y entidad de dichas circunstancias.
    Si hay dos o más circunstancias agravantes y ninguna atenuante, puede aplicar la pena superior en un grado.
    En el caso de concurrir circunstancias atenuantes y agravantes, se hará su compensación racional para la aplicación de la pena, graduando el valor de unas y otras.




    ART. 68.

    Cuando la pena señalada por la ley consta de dos o más grados, bien sea que los formen una o dos penas indivisibles y uno o más grados de otra divisible, o diversos grados de penas divisibles, el tribunal al aplicarla podrá recorrer toda su extensión, si no concurren en el hecho circunstancias atenuantes ni agravantes.
    Habiendo una sola circunstancia atenuante o una sola circunstancia agravante, no aplicará en el primer caso el grado máximo ni en el segundo el mínimo.
    Si son dos o más las circunstancias atenuantes y no hay ninguna agravante, el tribunal podrá imponer la pena inferior en uno, dos o tres grados al mínimo de los señalados por la ley, según sea el número y entidad de dichas circunstancias.
    Cuando, no concurriendo circunstancias atenuantes, hay dos o más agravantes, podrá imponer la inmediatamente superior en grado al máximo de los designados por la ley.
    Concurriendo circunstancias atenuantes y agravantes, se observará lo prescrito en los artículos anteriores para casos análogos.



    ART. 68 BIS.

    Sin perjuicio de lo dispuesto en los cuatro artículos anteriores, cuando sólo concurra una atenuante muy calificada el Tribunal podrá imponer la pena inferior en un grado al mínimo de la señalada al delito.

    ART. 69.

    Dentro de los límites de cada grado el tribunal determinará la cuantía de la pena en atención al número y entidad de las circunstancias atenuantes y agravantes y a la mayor o menor extensión del mal producido por el delito.


    ART. 70.

    En la aplicación de las multas el tribunal podrá recorrer toda la extensión en que la ley le permite imponerlas, consultando para determinar en cada caso su cuantía, no solo las circunstancias atenuantes y agravantes del hecho, sino principalmente el caudal o facultades del culpable. Asimismo, en casos calificados, de no concurrir agravantes y considerando las circunstancias anteriores, el juez podrá imponer una multa inferior al monto señalado en la ley, lo que deberá fundamentar en la sentencia.
    Tanto en la sentencia como en su ejecución el Tribunal podrá, atendidas las circunstancias, autorizar al afectado para pagar las multas por parcialidades, dentro de un límite que no exceda del plazo de un año. El no pago de una sola de las parcialidades, hará exigible el total de la multa adeudada.




    ART. 71.

    Cuando no concurran todos los requisitos que se exigen en el caso del núm. 8.° del art. 10 para eximir de responsabilidad, se observará lo dispuesto en el art. 490.



    ART. 72.

    En los casos en que aparezcan responsables en un mismo delito individuos mayores de dieciocho años y menores de esa edad, se aplicará a los mayores la pena que les habría correspondido sin esta circunstancia, aumentada en un grado, si éstos se hubieren prevalido de los menores en la perpetración del delito, pudiendo esta circunstancia ser apreciada en conciencia por el juez.




    ART. 73.

    Se aplicará asimismo la pena inferior en uno, dos o tres grados al mínimo de los señalados por la ley, cuando el hecho no fuere del todo excusable por falta de alguno de los requisitos que se exigen para eximir de responsabilidad criminal en los respectivos casos de que trata el art. 10, siempre que concurra el mayor número de ellos, imponiéndola en el grado que el tribunal estime correspondiente, atendido el número entidad de los requisitos que falten o concurran.
    Esta disposición se entiende sin perjuicio de la contenida en el art. 71.


    ART. 74.

    Al culpable de dos o más delitos se le impondrán todas las penas correspondientes a las diversas infracciones.
    El sentenciado cumplirá todas sus condenas simultáneamente, siendo posible. Cuando no lo fuere, o si de ello hubiere de resultar ilusoria alguna de las penas, las sufrirá en orden sucesivo, principiando por las más graves o sea las más altas en la escala respectiva, excepto las de confinamiento, extrañamiento, relegación y destierro, las cuales se ejecutarán después de haber cumplido cualquiera otra pena de las comprendidas en la escala gradual núm. 1.

